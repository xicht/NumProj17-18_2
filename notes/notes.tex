\documentclass[a4paper]{article}

\usepackage[ngerman]{babel}
\usepackage[utf8]{inputenc}
\usepackage[T1]{fontenc}

\usepackage{amsmath}
\usepackage{amssymb}

\usepackage{geometry}
\setlength{\parindent}{0pt}
\setlength{\parskip}{8pt}
\usepackage{enumerate}


%%%  AMSTHM
\usepackage{amsthm}
\newtheorem{satz}{Satz}
\newtheorem{lemma}[satz]{Lemma}
\theoremstyle{definition}
\newtheorem{defi}[satz]{Definition}
\newtheorem{bsp}[satz]{Beispiel}
\renewenvironment{proof}{\text{\textit{Beweis.}}}{\qed}
\renewcommand{\qedsymbol}{$\blacksquare$}


%%  MACROS MATHE
\newcommand{\C}{\mathbb{C}}
\newcommand{\N}{\mathbb{N}}
\newcommand{\Q}{\mathbb{Q}}
\newcommand{\R}{\mathbb{R}}
\newcommand{\Z}{\mathbb{Z}}
\newcommand{\U}{\mathcal{U}}
\newcommand{\T}{\mathcal{T}}
\newcommand{\OO}{\mathcal{O}}
\newcommand{\F}{\mathcal{F}}
\newcommand{\B}{\mathcal{B}}
\newcommand{\PM}[1]{\mathcal{P}(#1)}
\newcommand{\gdw}{\;\Longleftrightarrow\;}
\newcommand{\ve}[1]{{\emph{\textbf{#1}}}}
\newcommand{\norm}[2][]{\left\|#2\right\|_{#1}}


\newcommand{\seq}[2]{\left({#1}_{#2}\right)_{#2 \in \N}}
\newcommand{\seqq}[3]{\left({#1}_{{#2}_{#3}}\right)_{#3 \in \N}}
\newcommand{\seqqq}[4]{\left({#1}_{{{#2}_{#3}}_{#4}}\right)_{#4 \in \N}}

\newcommand{\dist}[2]{\textup{dist}\left(#1,#2\right)}

\renewcommand{\epsilon}{\varepsilon}
\renewcommand{\subset}{\subseteq}
\renewcommand{\supset}{\supseteq}

\begin{document}
\section{Funktionen zum Testen}

Sei $F(x,y) := a \cdot \sin \Big(c \big( \sin(x) -y \big)\Big) $

Dann gilt:
\begin{align*}
F(x,y)=0 &\iff \sin \Big(c \big( \sin(x) -y \big)\Big) = 0 \\
&\iff \exists k \in \Z :  c \big( \sin(x) -y \big) = k \pi \\
&\iff \exists k \in \Z :   \sin(x) -y  = \frac{k}{c} \pi \\
&\iff \exists k \in \Z :    y  = \sin(x) -\frac{k}{c} \pi \\
\end{align*}
Also ist die Nullstellenmenge Kopien des Graphen von $\sin(x)$ im Abstand von $\frac{k}{c} \pi$ an der $y$-Achse.

Außerdem ist
\[
\frac{\partial F}{\partial y}(x,y) =  -c\cdot a \cdot\cos \Big(c \big( \sin(x) -y \big)\Big)
\]
und analog
\begin{align*}
\frac{\partial F}{\partial y}(x,y) =0 \iff \exists k \in \Z :    y  = \sin(x) -\frac{2k+1}{2c} \pi \\
\end{align*}
und 
\[
\frac{\partial F}{\partial x}(x,y) =  c \cdot a \cdot\cos \Big(c \big( \sin(x) -y \big)\Big) \cos(x)
\]
mit
\begin{align*}
\frac{\partial F}{\partial x}(x,y) =0 \iff \exists k \in \Z :    y  = \sin(x) -\frac{2k+1}{2c} \pi \;\lor\;  y= \frac{2k+1}{2} \pi\\
\end{align*}
Also sind die Vorraussetzungen an die Ableitung leider nicht immer erfüllt



\end{document}



