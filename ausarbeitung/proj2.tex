\documentclass[a4paper,11pt,bibliography=totoc,listof=totoc,headinclude=true,cleardoublepage=empty,oneside]{scrartcl}
% Option "oneside" für einseitigen Druck. Weglassen, falls die Arbeit doppelseitig gedruckt wird

\usepackage[english,ngerman]{babel}
\usepackage[utf8]{inputenc}
%\usepackage{fullpage}
\usepackage{ifthen}
\usepackage{color}
\usepackage{amsmath,amsthm,amssymb,amsfonts}
\usepackage{graphicx}
\usepackage{psfrag}
\usepackage{float}
\usepackage{blindtext}
\usepackage{listings}
\usepackage{lineno}
\usepackage{tabularx}

% links in pdf
\usepackage[unicode,colorlinks=true,pagebackref=false]{hyperref}

% Zum Druck verwende schwarze Links!
%\usepackage[unicode,colorlinks=true,linkcolor=black,citecolor=black,urlcolor=black,pagebackref=false]{hyperref} 
	% colorlinks=false umrahmt Links statt einzufaerben, 


% document style
\KOMAoptions{footinclude=false} % Fusszeile wird nicht zu Satzspiegel gezaehlt
\KOMAoptions{headsepline=true} % Trennlinie zwischen Kopfzeile und Text
\KOMAoptions{DIV=12} % beeinflusst Satzspiegel
%\KOMAoptions{BCOR=8mm} % Bindekorrektur
\pagestyle{headings} % mit Kopfzeilen

\recalctypearea % berechne Satzspiegel neu

\definecolor{change}{rgb}{0,.55,.55}

\def\revision#1{{\color{red}#1}}


\setlength{\parindent}{0pt}
\setlength{\parskip}{5pt}

%Belegung einzelner Buchstaben
\newcommand{\A}{\mathcal{A}}
\newcommand{\B}{\mathcal{B}}
%\newcommand{\C}{\mathbb{C}}
\newcommand{\E}{\mathcal{E}}
\newcommand{\F}{\mathcal{F}}
\newcommand{\K}{\mathbb{K}}
\newcommand{\N}{\mathbb{N}}
\newcommand{\OO}{\mathcal{O}}
\newcommand{\Q}{\mathbb{Q}}
\newcommand{\R}{\mathbb{R}}
\newcommand{\T}{\mathsf{T}}


\begin{document}

%%%%%%%%%%%%%%%%%%%%%%%%%%%%%%%%%%%%%%%%%%%%%%%%%%%%%%%%%%%%%%%%%%%%%%%%%%%%%%%%%%%%%%%%%%%%%%%%%%%%%%%%%%%%%%
% TITELSEITE [OBLIGATORISCH]
%%%%%%%%%%%%%%%%%%%%%%%%%%%%%%%%%%%%%%%%%%%%%%%%%%%%%%%%%%%%%%%%%%%%%%%%%%%%%%%%%%%%%%%%%%%%%%%%%%%%%%%%%%%%%%

\pagenumbering{Alph}
\selectlanguage{ngerman}

\begin{titlepage}
  %\vspace*{-2cm}
  \begin{center}
    \includegraphics[width=0.45\textwidth]{TULogo.eps}
    \vskip 1cm%
    {\LARGE N~\Large U~M~E~R~I~K~P~R~O~J~E~K~T}
    \vskip 8mm
    {\huge\bfseries\color{change}Titel \\[1ex] ggf.\ mehrzeilig}
    \vskip 1cm
    \large 
    ausgef\"uhrt am    
    \vskip 0.75cm
    {\Large Institut f\"ur\\[1ex] Analysis und Scientific Computing}\\[1ex]
    {\Large TU Wien}
    \vskip0.75cm
    unter der Anleitung von
    \vskip0.75cm
    {\Large\bfseries\color{change}Name des Betreuers}\\[1ex]
    \vskip 0.5cm
    durch
    \vskip 0.5cm
    {\color{change} \Large\bfseries Markus Rinke }\\[1ex]
    Matrikelnummer: { \color{change} 1402581}\\[1ex]
    {\Large\bfseries Stefan Schrott}\\[1ex]
    Matrikelnummer: {1607388}\\[1ex]
   
  \end{center}
  
  \vfill
  
  \small
  Wien, am \today
  \vspace*{-15mm}
\end{titlepage}

\cleardoublepage

%%%%%%%%%%%%%%%%%%%%%%%%%%%%%%%%%%%%%%%%%%%%%%%%%%%%%%%%%%%%%%%%%%%%%%%%%%%%%%%%%%%%%%%%%%%%%%%%%%%%%%%%%%%%%%
% INHALTSVERZEICHNIS [OBLIGATORISCH]
%%%%%%%%%%%%%%%%%%%%%%%%%%%%%%%%%%%%%%%%%%%%%%%%%%%%%%%%%%%%%%%%%%%%%%%%%%%%%%%%%%%%%%%%%%%%%%%%%%%%%%%%%%%%%%

\pagenumbering{roman}
\selectlanguage{ngerman} 

\tableofcontents

\cleardoublepage
\pagenumbering{arabic} 


\section{Grundlagen}
Die Grundlage für die folgenden Überlegung ist der Hauptsatz über implizite Funktionen im Spezialfall von Funktionen $F: \R \times \R \to \R$. 

\end{document}
