\documentclass[a4paper,11pt,bibliography=totoc,listof=totoc,headinclude=true,cleardoublepage=empty,oneside]{scrartcl}
% Option "oneside" für einseitigen Druck. Weglassen, falls die Arbeit doppelseitig gedruckt wird

\usepackage[english,ngerman]{babel}
\usepackage[utf8]{inputenc}
%\usepackage{fullpage}
\usepackage{ifthen}
\usepackage{color}
\usepackage{amsmath,amsthm,amssymb,amsfonts}
\usepackage{graphicx}
\usepackage{psfrag}
\usepackage{float}
\usepackage{blindtext}
\usepackage{listings}
\usepackage{lineno}
\usepackage{tabularx}

% links in pdf
\usepackage[unicode,colorlinks=true,pagebackref=false]{hyperref}

% Zum Druck verwende schwarze Links!
%\usepackage[unicode,colorlinks=true,linkcolor=black,citecolor=black,urlcolor=black,pagebackref=false]{hyperref} 
	% colorlinks=false umrahmt Links statt einzufaerben, 


% document style
\KOMAoptions{footinclude=false} % Fusszeile wird nicht zu Satzspiegel gezaehlt
\KOMAoptions{headsepline=true} % Trennlinie zwischen Kopfzeile und Text
\KOMAoptions{DIV=12} % beeinflusst Satzspiegel
%\KOMAoptions{BCOR=8mm} % Bindekorrektur
\pagestyle{headings} % mit Kopfzeilen

\recalctypearea % berechne Satzspiegel neu

\definecolor{change}{rgb}{0,.55,.55}

\def\revision#1{{\color{red}#1}}


\setlength{\parindent}{0pt}
\setlength{\parskip}{5pt}

%Belegung einzelner Buchstaben
\newcommand{\A}{\mathcal{A}}
\newcommand{\B}{\mathcal{B}}
%\newcommand{\C}{\mathbb{C}}
\newcommand{\E}{\mathcal{E}}
\newcommand{\F}{\mathcal{F}}
\newcommand{\K}{\mathbb{K}}
\newcommand{\N}{\mathbb{N}}
\newcommand{\OO}{\mathcal{O}}
\newcommand{\Q}{\mathbb{Q}}
\newcommand{\R}{\mathbb{R}}
\newcommand{\T}{\mathsf{T}}


%sonstiges zeug
\renewcommand{\subset}{\subseteq}
\renewcommand{\supset}{\supseteq}
\renewcommand{\epsilon}{\varepsilon}

%markos
\newcommand{\diff}[2]{\frac{\partial #1}{\partial #2}}
\newcommand{\secdiff}[2]{\frac{\partial^2 #1}{\partial^2 #2}}
\newcommand{\diffdiff}[3]{\frac{\partial^2 #1}{\partial #2 \partial #3}}

\begin{document}

%%%%%%%%%%%%%%%%%%%%%%%%%%%%%%%%%%%%%%%%%%%%%%%%%%%%%%%%%%%%%%%%%%%%%%%%%%%%%%%%%%%%%%%%%%%%%%%%%%%%%%%%%%%%%%
% TITELSEITE [OBLIGATORISCH]
%%%%%%%%%%%%%%%%%%%%%%%%%%%%%%%%%%%%%%%%%%%%%%%%%%%%%%%%%%%%%%%%%%%%%%%%%%%%%%%%%%%%%%%%%%%%%%%%%%%%%%%%%%%%%%

\pagenumbering{Alph}
\selectlanguage{ngerman}

\begin{titlepage}
  %\vspace*{-2cm}
  \begin{center}
    \includegraphics[width=0.45\textwidth]{TULogo.eps}
    \vskip 1cm%
    {\LARGE N~\Large U~M~E~R~I~K~P~R~O~J~E~K~T}
    \vskip 8mm
    {\huge\bfseries\color{change}Titel \\[1ex] ggf.\ mehrzeilig}
    \vskip 1cm
    \large 
    ausgef\"uhrt am    
    \vskip 0.75cm
    {\Large Institut f\"ur\\[1ex] Analysis und Scientific Computing}\\[1ex]
    {\Large TU Wien}
    \vskip0.75cm
    unter der Anleitung von
    \vskip0.75cm
    {\Large\bfseries\color{change}Name des Betreuers}\\[1ex]
    \vskip 0.5cm
    durch
    \vskip 0.5cm
    {\color{change} \Large\bfseries Markus Rinke }\\[1ex]
    Matrikelnummer: { \color{change} 1402581}\\[1ex]
    {\Large\bfseries Stefan Schrott}\\[1ex]
    Matrikelnummer: {1607388}\\[1ex]
   
  \end{center}
  
  \vfill
  
  \small
  Wien, am \today
  \vspace*{-15mm}
\end{titlepage}

\cleardoublepage

%%%%%%%%%%%%%%%%%%%%%%%%%%%%%%%%%%%%%%%%%%%%%%%%%%%%%%%%%%%%%%%%%%%%%%%%%%%%%%%%%%%%%%%%%%%%%%%%%%%%%%%%%%%%%%
% INHALTSVERZEICHNIS [OBLIGATORISCH]
%%%%%%%%%%%%%%%%%%%%%%%%%%%%%%%%%%%%%%%%%%%%%%%%%%%%%%%%%%%%%%%%%%%%%%%%%%%%%%%%%%%%%%%%%%%%%%%%%%%%%%%%%%%%%%

\pagenumbering{roman}
\selectlanguage{ngerman} 

\tableofcontents

\cleardoublepage
\pagenumbering{arabic} 


\section{Grundlagen}
Die Grundlage für die folgenden Überlegung ist der Hauptsatz über implizite Funktionen im Spezialfall von Funktionen $F: A \times B \to \R$, wobei $A$ und $B$ der Einfachheit halber offene Intervalle seien.

\textbf{Satz:} Seien $a<b$ sowie $c<d \in \R$ und $F : (a,b) \times (c,d) \to \R$ stetig differenzierbar. Seien $x_0 \in (a,b)$ und $y_0 \in (c,d)$, sodass $F(x_0,y_0)=0$ und $\diff{F}{y}(x_0,y_0) \neq 0$. 

Dann existieren $a_0,b_0 \in \R$ mit $a<a_0<x_0<b_0<b$ und eine stetig differenzierbare Funktion $f : (a_0,b_0) \to \R$ mit $f(x_0)=y_0$, sodass 
\[
\forall x \in (a_0,b_0) : F(x,f(x))=0
\]
und
\begin{align}\label{eq:fstrich}
\forall x \in (a_0,b_0) :  f'(x) = - \frac{\diff{F}{x}(x,f(x))}{\diff{F}{y}(x,f(x))}.
\end{align}

\textbf{Beweis:} Unter den gegebenen Voraussetzungen ist der Hauptsatz über implizite Funktionen anwendbar und liefert Umgebungen $U$ von $x_0$ und $V$ von $y_0$ und eine Funktion $f: U \to V$ mit den geforderten Eigenschaften. Da $x_0$ ein innere Punkt von $U$ ist, enthält $U$ ein Intervall $(a_0,b_0)$ mit den geforderten Eigenschaften.

Die Umgebung $V \subset \R$ in der Zielmenge von $f$ kann durch ganz $\R$ ersetzt werden, da wir nur behauptet haben, dass $y=f(x)$ eine Lösung von $F(x,\:\cdot \:)=0$ ist, allerdings nicht dass diese eindeutig ist. \hfill
$\blacksquare$

\textbf{Satz:} Sei unter den Vorraussetzungen des vorherigen Satz $F$ zwei mal stetig differenzierbar.

Dann ist $f \in C^2((a_0,b_0))$ mit $f''(x) = $
\[
\frac{ \!-\secdiff{F}{x}(x,f(x)) \left( \diff{F}{y}(x,f(x)) \right)^2 \!\!\! +\! 2 \diffdiff{F}{x}{y}(x,f(x))\diff{F}{x}(x,f(x))\diff{F}{y}(x,f(x)) \!-\! \secdiff{F}{y}(x,f(x)) \left( \diff{F}{x}(x,f(x)) \right)^2  }{ \left(\diff{F}{y}(x,f(x))\right)^3 }.
\]

Außerdem gilt:
\[
\forall x \in (a_0,b_0) \exists \xi \in (x_0,x)\cup (x,x_0) : f(x) = y_0 + \frac{\diff{F}{x}(x_0,y_0)}{\diff{F}{y}(x_0,y_0)}(x-x_0) + \frac{f''(\xi)}{2}(x-x_0)^2.
\]

\textbf{Beweis:} Aus $F \in C^2$ folgt mit der Kettenregel und Einsetzen der Darstellung \eqref{eq:fstrich}für $f'$:
\begin{align*}
\frac{d}{dx}\left( \diff{F}{x}(x,f(x))  \right) &= \left( \secdiff{F}{x}(x,f(x)) , \diffdiff{F}{x}{y}(x,f(x))  \right) \cdot \binom{1}{f'(x)} \\
&= \secdiff{F}{x}(x,f(x)) - \diffdiff{F}{x}{y}(x,f(x)) \frac{\diff{F}{x}(x,f(x))}{\diff{F}{y}(x,f(x))}.
\end{align*}
Für $\frac{d}{dx}\left( \diff{F}{y}(x,f(x))  \right)$ erhält man analog eine ähnliche Darstellung. Damit kann man den Ausdruck \eqref{eq:fstrich} mithilfe der Quotientenregel differenzieren und erhält durch Erweitern mit $\diff{F}{y}(x,f(x))$ obige Darstellung für $f''$.

Die zweite Aussage folgt aus dem Satz von Taylor. \hfill $\blacksquare$

\end{document}
